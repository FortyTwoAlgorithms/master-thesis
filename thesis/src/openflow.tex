\section{OpenFlow}



\subsection{Definição}

Em 2008 o protocolo OpenFlow foi publicado. Ele permitiu que pesquisadores
pudessem criar experimentos com novos protocolos em redes convencionais
\citep{nick2008openflow}.
O OpenFlow foi criado como um padrão aberto, o que permite que todos os
fabricantes de equipamentos de redes possam habilitar seus produtos a esse
padrão.

O protocolo consiste em uma interface de programação para o comutador.
Assim, um programador pode, através de um programa, controlar a forma como
um comutador executa seu encaminhamento de pacotes.
De uma maneira bem clara, o protocolo OpenFlow separa o plano de dados
do plano de controle, fazendo com que soluções SDN possam ser criadas
e experimentadas.
Por ser uma solução de baixo custo, o OpenFlow obteve boa aceitação na
academia e no mercado, dado o volume de empresas e pesquisas relacionadas
ou que utilizam o protocolo.

\subsection{Componentes}

Na arquitetura estabelecida pelo protocolo OpenFlow existem dois papéis
principais.
O controlador e o comutador OpenFlow.
Uma separação baseada no modelo das Redes definidas por software (SDN)
conforme pode ser visto na figura~\ref{fig:controller-secure-switch}.

\begin{figure}[h!]
    \centering
    \label{fig:controller-secure-switch}
    \includegraphics[width=\linewidth]{img/controller-secure-switch}
    \caption{Divisão da arquitetura OpenFlow}
\end{figure}

\subsection{Arquitetura do Comutador}

O comutador OpenFlow pode ser dividido em três principais partes.
A primeira é o canal de comunicação seguro com o controlador.
A segunda é a interface do protocolo OpenFlow que permite ao controlador
controlar o comutador.
A terceira é sua tabela de fluxos.

O canal de comunicação seguro tem a garantia de confiabilidade na troca
de mensagens entre o controlador e o comutador através do protocolo SSL
(Secure Socket Layer).
Isso adiciona proteção à rede em relação a ataques de elementos mal
intencionados \citep{rothenberg2010openflow}.

A interface do protocolo OpenFlow é a padronização das mensagens enviadas
pelo controlador ao comutador de modo a definir o comportamento do encaminhamento
de pacotes.
Ou seja, é o conjunto de possíveis instruções para se controlar de maneira
genérica o plano de dados de qualquer \emph{hardware} de redes habilitado
ao padrão OpenFlow.

A tabela de fluxos é composta por regras.
Cada regra consiste em ações associadas à fluxos.
Através dessa tabela o comutador executa o encaminhamento de pacotes.
As entradas dessa tabela são atualizadas pelo controlador.

\begin{figure}[h!]
    \centering
    \label{fig:switch-arch}
    \includegraphics[width=\linewidth]{img/switch-architecture}
    \caption{Arquitetura do comutador OpenFlow}
\end{figure}


\subsection{Fluxos}

Um fluxo é a representação de um ou mais pacotes em função de suas
características.
Essas características variam de acordo com os campos que definem um fluxo.
Os fluxos são genéricos.
Um fluxo pode caracterizar pacotes que ainda não passaram pelo comutador mas
que possuem as mesmas características que compõem aquele fluxo.
Sendo assim, a ação associada ao fluxo é aplicada a esses novos pacotes.

\input{tbl/flow-table}


A tabela de fluxos dentro do comutador OpenFlow identifica os fluxos para que o
plano de dados execute ações sobre os pacotes que são pertencentes àquele
fluxo.
A tabela \ref{tbl:flowtable} apresenta de maneira simplificada a tabela
de fluxos dentro do comutador OpenFlow.

\subsection{Campos para definição de fluxos}

Os campos que definem fluxos se estendem da camada 1 até a camada 4 da
pilha TCP/IP.
A figura \ref{fig:of-header} apresenta os campos e suas respectivas camadas.

\begin{figure}[h!]
    \centering
    \label{fig:of-header}
    \includegraphics[width=\linewidth]{img/openflow-header}
    \caption{Campos para definição de fluxos OpenFlow}
\end{figure}

Quando um novo pacote ingressa no comutador OpenFlow esses campos são
preenchido e encaminhado para o controlador.
Através da análise dessas informações do fluxo o controlador envia uma
mensagem ao comutador instalando/atualizando regras na tabela de fluxos.

Esse mecanismo de rotulagem e identificação de tráfego e pacotes (fluxos) é
inspirado no MPLS (\emph{Multi-protocol Label Switching})
\citep{bruce2008mpls}.

\subsection{Ações}

As ações (\emph{actions}) são aplicadas aos fluxos especificados na tabela
de fluxos do comutador.
Ações como encaminhamento, atualização de cabeçalho e rejeição podem ser
aplicadas aos pacotes pertencentes àquele fluxo.

A especificação do protocolo OpenFlow \citep{ofprotocol2015} descreve todas
as possíveis ações que podem ser aplicadas.
A lista abaixo apresenta algumas ações:

\begin{multicols}{3}
    \begin{itemize}
        \item Forwarding
        \item Drop
        \item Set
        \item Strip
        \item Copy-in
        \item Copy-out
        \item Push
        \item Pop
        \item Dec
    \end{itemize}
\end{multicols}

Essas ações são aplicadas a cada novo pacote pertencente ao algum determinado
fluxo pertencente à tabela de fluxos.

\subsection{Controlador}

O controlador é a entidade na rede que representa o plano de controle dentro
da arquitetura OpenFlow.
Apesar de ser um entidade logicamente centralizada, ele pode ser distribuído.
O controlador é um software que se conecta de maneira segura ao comutador
OpenFlow.
Através da interface de programação do protocolo o controlador manipula
a tabela de fluxos do comutador.

Em função dessa arquitetura, é possível ter visão e controle de estado global
da rede.
Como o protocolo OpenFlow especifica que o comutador armazena alguns dados
estatísticos, por exemplo de portas e fluxos, o controlador pode tomar
decisões ou executar análises sobre a rede baseando-se nessas estatísticas.

O controlador exerce um papel importante para os serviços em execução dentro
da rede.
Outros servidores rodando aplicações podem fazer requisições e troca de
mensagens com o controlador da rede, de modo a obter informações topológicas
ou de estado da rede.

Com o plano de controle isolado em uma aplicação logicamente centralizada,
um programador tem de lidar com problemas típicos de desenvolvimento de
software e sistemas distribuídos como:

\begin{itemize}
    \item Tolerância a falhas
    \item Persistência
    \item Eficiência
    \item Design de implementação
    \item Debugging
    \item Testes
\end{itemize}

Existem vários softwares controladores \emph{open source} no mercado que
podem ser utilizados em produção em para desenvolvimento de pesquisas:

\begin{itemize}
    \item Biblioteca \href{http://opennetworkingfoundation.github.io/libfluid/index.html}{Libfluid}
        para criação de aplicações/controladores em SDN \citep{libfluid2015}
    \item \href{http://www.noxrepo.org/nox/about-nox/}{Nox Controller}
        \citep{nox2015}
    \item \href{https://openflow.stanford.edu/display/Beacon/Home}{Beacon}
        \citep{beacon2015}
    \item \href{http://www.noxrepo.org/pox/about-pox/}{Pox Controller}
        \citep{pox2015}
    \item \href{http://osrg.github.io/ryu/}{Ryu} \citep{ryu2015}
\end{itemize}

